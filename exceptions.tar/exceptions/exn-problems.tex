\documentclass{article}

\usepackage{amsmath}
\usepackage{amssymb}
\usepackage{stmaryrd}
\usepackage{mathtools}
\usepackage{color}
\usepackage{mathpartir}
\usepackage{fullpage}

\newcommand{\btrue}{\mathit{true}}
\newcommand{\bfalse}{\mathit{false}}

%%%%%%%%%%%%%%%%%%%%%%%%%%%%%%%%%%% PROC %%%%%%%%%%%%%%%%%%%%%%%%%%%%%%%%%%%
\newcommand{\pdone}{\mathit{done}}
\newcommand{\pmake}[2]{\mathit{mk}~#1 \{  #2 \}}
\newcommand{\psend}[3]{#1 \Rightarrow #2; #3}
\newcommand{\precv}[3]{#1 \Leftarrow #2; #3}
\newcommand{\pcond}[2]{\left[ #1 \right] #2}
\newcommand{\pplus}[2]{#1 + #2}
\newcommand{\pparl}[2]{#1 | #2}

\newcommand{\ltran}[1]{\overset{#1}{\longrightarrow}}
\newcommand{\sendt}[2]{\overset{#1, #2}{\longrightarrow}}
\newcommand{\recvt}[2]{\overset{\bar{#1}, #2}{\longrightarrow}}

\title{Takehome Final, Written Problems}
\author{CS 538, Spring 2020}

\begin{document}

\maketitle

\newcommand*{\Tr}{\mathit{true}}
\newcommand*{\Fa}{\mathit{false}}

In this problem, we will be adding basic exceptions to the core functional
language we saw in the first half of the course. The core language is a lambda
calculus with booleans, integers, and strings:
%
\begin{align*}
  \text{Num } n &::= \mathbb{N}
  \\
  \text{Str } s &::= \text{``letters''}
  \\
  \text{Val } v &::= x
    \mid \lambda x.~e
    \mid \Tr
    \mid \Fa
    \mid n
    \mid s
  \\
  \text{Exp } e &::= x
    \mid \lambda x.~e
    \mid e_1~e_2
    \mid \Tr
    \mid \Fa
    \mid n
    \mid s
    \mid \mathit{if}~e_1~\mathit{then}~e_2~\mathit{else}~e_3
    \mid e_1 = e_2
    \mid e_1 + e_2
\end{align*}
%
In words, we have the following components:
%
\begin{itemize}
  \item Numbers $n$ ($0$, $1$, \dots)
  \item Strings $s$ (``foo'', ``bar'', \dots)
  \item Variables $x$
  \item Values $v$ ($42$, ``foo'', $\lambda x.~x + 1$, \dots)
  \item Expressions $e$ ($x + 0$, $3 \times 5$, \dots)
\end{itemize}
%
The language is eager (call-by-value), with the following operational semantics:
%
\begin{mathpar}
  \inferrule
  { e_1 \to e_1' }
  { e_1~e_2 \to e_1'~e_2 }
  \and
  \inferrule
  { e_2 \to e_2' }
  { (\lambda x.~e_1)~e_2 \to (\lambda x.~e_1)~e_2' }
  \and
  \inferrule
  { }
  { (\lambda x.~e)~v \to e[x \mapsto v] }
  \\
{
  \inferrule
  { e_1 \to e_1' }
  { \mathit{if}~e_1~\mathit{then}~e_2~\mathit{else}~e_3
  \to \mathit{if}~e_1'~\mathit{then}~e_2~\mathit{else}~e_3 }
  \and
  \inferrule
  {~}
  { \mathit{if}~\Tr~\mathit{then}~e_2~\mathit{else}~e_3 \to e_2 }
  \and
  \inferrule
  {~}
  { \mathit{if}~\Fa~\mathit{then}~e_2~\mathit{else}~e_3 \to e_3 }
}
  \\
  \inferrule
  { e_1 \to e_1' }
  { e_1 = e_2 \to e_1' = e_2 }
  \and
  \inferrule
  { e_2 \to e_2' }
  { v_1 = e_2 \to v_1 = e_2' }
  \and
  \inferrule
  { v_1 \text{ is equal to } v_2 }
  { v_1 = v_2 \to \Tr }
  \and
  \inferrule
  { v_1 \text{ is not equal to } v_2 }
  { v_1 = v_2 \to \Fa }
  \\
  \inferrule
  { e_1 \to e_1' }
  { e_1 + e_2 \to e_1' + e_2 }
  \and
  \inferrule
  { e_2 \to e_2' }
  { v_1 + e_2 \to v_1 + e_2' }
  \and
  \inferrule
  { n = n_1 \text{ plus } n_2 }
  { n_1 + n_2 \to n }
\end{mathpar}

\section{Raising exceptions}

To raise exceptions, we add a new expression:
%
\[
  \text{Exp } e ::= \mathit{raise}(s) \mid \cdots
\]
%
The expression $\mathit{raise}(s)$ is an exception, and the string $s$ describes
the kind of exception. If we hit an exception during evaluation, then we don't
evaluate the rest of the expression---we just step to the exception. We can
model this behavior using the following operational rules:
%
\begin{mathpar}
  \inferrule
  {~}
  { \mathit{raise}(s)~e \to \mathit{raise}(s) } 
  \and
  \inferrule
  { }
  { (\lambda x.~e)~\mathit{raise}(s) \to \mathit{raise}(s) }
  \and
  \inferrule
  {~}
  { \mathit{if}~\mathit{raise}(s)~\mathit{then}~e_2~\mathit{else}~e_3 \to \mathit{raise}(s) }
  \and
  \inferrule
  {~}
  { \mathit{raise}(s) = e \to \mathit{raise}(s) }
  \and
  \inferrule
  {~}
  { v = \mathit{raise}(s) \to \mathit{raise}(s) }
\end{mathpar}
%
Answer the following questions.
%
\begin{enumerate}
  \item Write down similar rules for raising exceptions from $e_1 + e_2$. Hint:
    you should have two new rules, almost the same as the new rules for equality.
  \item Write a (lambda calculus) function that takes two arguments $x$ and $y$.
    If $x$ is equal to $42$ then raise the exception
    $\mathit{raise}(\text{``1st 42''})$, otherwise if $y$ is equal to $21$ then
    raise the exception $\mathit{raise}(\text{``2nd 21''})$, otherwise return
    the sum of $x$ and $y$.
\end{enumerate}

\section{Handling exceptions}

Most languages with exceptions also have exception handlers, which are used to
recover and run error-handling code when an exception is raised. We add the
following expression to our language:
%
\[
  \text{Exp } e ::= \mathit{try}~e_1~\mathit{with}~e_2 \mid \cdots
\]
%
The idea is that $e_1$ is a piece of code that may raise an exception, and $e_2$
is a function that takes a string describing the exception, and then does
something (handles it). If $e_1$ doesn't raise an exception, then $e_2$ is never
run.

To model this behavior, we add the following step rules:
%
\begin{mathpar}
  \inferrule
  { e_1 \to e_1' }
  { \mathit{try}~e_1~\mathit{with}~e_2 \to \mathit{try}~e_1'~\mathit{with}~e_2 } 
  \and
  \inferrule
  {~}
  { \mathit{try}~\mathit{raise}(s)~\mathit{with}~e \to e~s }
  \and
  \inferrule
  {~}
  { \mathit{try}~v~\mathit{with}~e \to v }
\end{mathpar}
%
We have carefully ensured that $\mathit{raise}(s)$ is \emph{not} a regular value
$v$, so the last two rules do not overlap.

Answer the following questions.
%
\begin{enumerate}
  \item Suppose that $\mathit{foo}$ is a piece of code. Write a program that
    runs $\mathit{foo}$, handles an exception if it is labeled $\text{``1st
    42''}$ by returning $0$, otherwise re-raises the exception. (Your answer
    should include ``$\mathit{foo}$'' somewhere. You do not need to show how
    this program steps.)
  \item Show how your previous answer steps when $\mathit{foo}$ is equal to
    $(\lambda x.~\mathit{if}~x = 0~\mathit{then}~\mathit{raise}(\text{``1st
    42''})~\mathit{else}~\mathit{raise}(\text{``2nd 21''}))~0$.
  \item Show how your previous answer steps when $\mathit{foo}$ is equal to
    $(\lambda x.~\mathit{if}~x = 0~\mathit{then}~\mathit{raise}(\text{``1st
    42''})~\mathit{else}~\mathit{raise}(\text{``2nd 21''}))~1$.
\end{enumerate}

\end{document}
